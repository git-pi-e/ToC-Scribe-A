\documentclass[12pt]{article}
\usepackage[left=3cm, right=3cm, top=2cm]{geometry}

%\usepackage[light,condensed,math]{anttor}
\usepackage[T1]{fontenc}
\usepackage{amsmath,graphicx}
\usepackage{hyperref}

%%cmark, xmark
\usepackage{amssymb}% http://ctan.org/pkg/amssymb
\usepackage{pifont}% http://ctan.org/pkg/pifont
\newcommand{\cmark}{\ding{51}}%
\newcommand{\xmark}{\ding{55}}%
\usepackage[table]{xcolor}% http://ctan.org/pkg/xcolor
\usepackage{tcolorbox}
\usepackage{float}
\usepackage{tikz}

\usetikzlibrary{arrows,automata,positioning}

%%cmark, xmark
\usepackage{amssymb}% http://ctan.org/pkg/amssymb
\usepackage{pifont}% http://ctan.org/pkg/pifont
% \newcommand{\cmark}{\ding{51}}%
% \newcommand{\xmark}{\ding{55}}%
%\usepackage[table]{xcolor}% http://ctan.org/pkg/xcolor
% \usepackage{tcolorbox}
\usepackage{float}

%
% The following macro is used to generate the header.
%
\newcommand{\lecture}[2]{
   \pagestyle{myheadings}
   \thispagestyle{plain}
   \newpage
   \noindent
   \begin{center}
   \framebox{
      \vbox{\vspace{2mm}
    \hbox to 6.28in { {\bf CS F351: Theory of Computation
	\hfill 16 Sep, 2022} }
       \vspace{4mm}
       \hbox to 6.28in { {\Large \hfill #1: #2  \hfill} }
       \vspace{2mm}
       \hbox to 6.28in { {\it \hfill Scribe: Piyush Mohite} }
      \vspace{2mm}}
   }
   \end{center}
   \vspace*{4mm}
}
%

\usepackage{amsthm}
\newtheorem{theorem}{Theorem}
\newtheorem{exercise}[theorem]{exercise}

\begin{document}
\lecture{8}{More Closure Properties}

\section*{More Closure Properties}
\subsection{Monoid}
A tuple $ (S, \cdot, e) $ is known as a \emph{monoid}, if:
\begin{itemize}
   \item[$-$] $ S $ is a set.
   \item[$-$]  $ \cdot : $ is \emph{associative}, i.e., $ \forall u,v,w \in S, (w \cdot u) \cdot v = w \cdot(u \cdot v) $
   \item[$-$] $e \in S $ is the identity element, i.e., $ \forall w \in S, e \cdot w = w \cdot e = w $
\end{itemize}

   Eg:  $ (\Sigma^\star, \cdot, e) $  is a monoid.



   Eg: $ (\mathbb{N}, +, 0) $ is a monoid. \hspace{2.05cm}  [Also commutative.]


   Eg: $ (\{e,a\}, \cdot, e) $ is \emph{NOT} a monoid. \hspace{0.1cm} [Because, if we consider $'a'$  and $'a'$, we get the concatenation $ a \cdot a = aa $, which is not a part of the set $\{e,a\}$, since it has to be $ S x S \rightarrow S $.]




\textbf{\underline{Specification:}} (trying to describe what is required in words)
\[ \exists x \exists y.( x < y \land R_a(x) \land R_b(y)) \]

\textbf{$R_a(x) $:} Tells us that at the position $'x'$ in the word, we have the letter $'a'$.


\textbf{$R_b(y) $:} Tells us that at the position $'y'$ in the word, we have the letter $'b'$.


\subsection{Homomorphism}
A function $ h : S_1 \rightarrow S_2 $ is called a (monoid) homomorphism from $ (S_1, \cdot_1, e_1) $ to $ (S_1, \cdot_1, e_1) $, if:

\begin{itemize}
   \item[$ - $] $ h(e_1) = e_2$.
   \item[$ - $] $ \forall u,v \in S_1, h(u \cdot_1  v) = h(u) \cdot_2 h(v) $.
\end{itemize}

Example: $ h : \{a,b{\}}^* \rightarrow \mathbb{N} $ is defined as $ h(w) = |w| $ is a monoid homomorphism from $ (\Sigma^*, \cdot, e) $\ to $(\mathbb{N}, +, 0) $.

\emph{Explanation: } $ h(e) = |e| = 0 $ and $ h(u \cdot v) = |u \cdot v| = |u| + |v| $.


It follows from these properties that any homomorphism defined on $\Sigma^*$ is uniquely determined by it's values on $\Sigma$.
Example: if $h(a) = ccc $ and $ h(b) = dd $, then \[ h(abaab) = h(a)h(a)h(b)h(a)h(b) = cccddccccccdd \]

If $A\subseteq\Sigma^*$, define \[ h(A) \stackrel{def}{=} \{h(x) \,|\, x \in A\}\]

and if $ B\subseteq\Gamma^*$, define \[ h^{-1}(B) \stackrel{def}{=} \{h(x) \,|\, x \in B\} \]
The set $h(A)$ is called the image of $A$ under $h$, and the set $h^{-1}(B)$ is called the preimage of $B$ under $h$.


\begin{theorem}
   Let h be a monoid homomorphism from $ (\Sigma^*, \cdot, e) $ to $ (\Gamma^*, \cdot, e) $. Then:
   \begin{enumerate}
      \item If $ L \subset \Sigma^* $ is regular, then $ h(L) \subseteq \Gamma^* $ is regular. \[ h(L) := \{ h(w) \, | \, w \in L \}\]
      \item If $ L \subseteq \Gamma^* $ is regular, then $ h^-1(L) \subseteq \Gamma^* $ is regular. \[ h^{-1}(L) := \{ w \in \Sigma^* \, | \, h(w)\in L \}\]
   \end{enumerate}

\end{theorem}

\emph{Explanation}: Any homomorphic image or homomorphic preimage of a regular language is regular.


\emph{\textbf{Example:}} \[ L_1 := \{e,aa,aaaa,aaaaaa,\dots,\} \]
\[h(L_1) := \{e,00,000,0000,\dots,\}\]
Here, if $L_1$ is regular, $h(L_1)$ is also regular.

\emph{\textbf{Example:}} \[ h^{-1}(0) = \{a,c\} \] \[\space h^{-1}(00) = \{aa,cc,ac,ca\}\]
Now, $h^{-1}(L_1) =$ set of all words of even length having $a$'s or $ c$'s.-

$\therefore$ $h^{-1}(L_1)$ is regular if $L_1$ is regular.


\end{document}
