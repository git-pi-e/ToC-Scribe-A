\documentclass[12pt]{article}
\usepackage[left=3cm, right=3cm, top=2cm]{geometry}

%\usepackage[light,condensed,math]{anttor}
\usepackage[T1]{fontenc}
\usepackage{amsmath,graphicx}
\usepackage{hyperref}


%%cmark, xmark
\usepackage{amssymb}% http://ctan.org/pkg/amssymb
\usepackage{pifont}% http://ctan.org/pkg/pifont
\newcommand{\cmark}{\ding{51}}%
\newcommand{\xmark}{\ding{55}}%
\usepackage[table]{xcolor}% http://ctan.org/pkg/xcolor
\usepackage{tcolorbox}
\usepackage{tikz}
\usepackage{float}

%
% The following macro is used to generate the header.
%
\newcommand{\lecture}[2]{
   \pagestyle{myheadings}
   \thispagestyle{plain}
   \newpage
   \noindent
   \begin{center}
   \framebox{
      \vbox{\vspace{2mm}
    \hbox to 6.28in { {\bf CS F351: Theory of Computation
	\hfill 16 Sep, 2022} }
       \vspace{4mm}
       \hbox to 6.28in { {\Large \hfill #1: #2  \hfill} }
       \vspace{2mm}
       \hbox to 6.28in { {\it \hfill Scribe: Piyush Mohite} }
      \vspace{2mm}}
   }
   \end{center}
   \vspace*{4mm}
}
%

\usepackage{amsthm}
\newtheorem{theorem}{Theorem}
\newtheorem{exercise}[theorem]{exercise}

\begin{document}
\lecture{8}{More Closure Properties}

\section*{More Closure Properties}
\newtheorem*{Monoid}{Hi}
A tuple $ (S, \bullet, e) $ is known as a \emph{monoid}, if:
\begin{itemize}
   \item[$-$] $ S $ is a set.
   \item[$-$]  $ \bullet : $ is \emph{associative}, i.e., $ \forall u,v,w \in S, (w \bullet u) \bullet v = w \bullet(u \bullet v) $
   \item[$-$] $e \in S $ is the identity element, i.e., $ \forall w \in S, e \bullet w = w \bullet e = w $
\end{itemize}

Eg: $ (\Sigma^\star, \bullet, e)$ is a monoid.


Eg: $ (\mathbb{N}, + , 0) $ is a monoid.   [Also commutative.]


Eg: $ (\{e,a\}, \bullet, e)$ is \emph{NOT} a monoid.
[Because, if we consider $'a'$  and $'a'$, we get the concatenation $a \bullet a = aa$, which is not a part of the set $\{e,a\}$, since it has to be $ S x S \rightarrow S $]






\end{document}
