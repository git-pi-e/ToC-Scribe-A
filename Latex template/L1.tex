\documentclass[12pt]{article}
\usepackage[left=3cm, right=3cm, top=2cm]{geometry}

%\usepackage[light,condensed,math]{anttor}
\usepackage[T1]{fontenc}
\usepackage{amsmath,graphicx}
\usepackage{hyperref}


%%cmark, xmark
\usepackage{amssymb}% http://ctan.org/pkg/amssymb
\usepackage{pifont}% http://ctan.org/pkg/pifont
\newcommand{\cmark}{\ding{51}}%
\newcommand{\xmark}{\ding{55}}%
\usepackage[table]{xcolor}% http://ctan.org/pkg/xcolor
\usepackage{tcolorbox}
\usepackage{tikz}
\usepackage{float}

%
% The following macro is used to generate the header.
%
\newcommand{\lecture}[2]{
   \pagestyle{myheadings}
   \thispagestyle{plain}
   \newpage
   \noindent
   \begin{center}
   \framebox{
      \vbox{\vspace{2mm}
    \hbox to 6.28in { {\bf CS F351: Theory of Computation
	\hfill Sep 2022} }
       \vspace{4mm}
       \hbox to 6.28in { {\Large \hfill #1: #2  \hfill} }
       \vspace{2mm}
       \hbox to 6.28in { {\it \hfill Scribe: Anup B Mathew} }
      \vspace{2mm}}
   }
   \end{center}
   \vspace*{4mm}
}
%

\usepackage{amsthm}
\newtheorem{theorem}{Theorem}
\newtheorem{exercise}[theorem]{exercise}

\begin{document}
\lecture{1}{Course Introduction}

The term \textit{computable} (or \textit{calculable}) has always been part of the scientific lexicon. Today we have a better understanding of this term due to the prevalence of computers; something is considered calculable, if a computer can do it. However it wasn't until the early 20th century (before the existence of computers) that there was a need to make this notion precise. The need for precision came from the following question asked by logicians of the time:  ``Do there exist mechanical rules that can identify mathematical truths and falsehoods?''.

In the 1930's there were many attempts to formalize the notion of computability.
\begin{itemize}
\item Kurt G\"odel (of the G\"odel imcompleteness theorem fame) proposed the system of \emph{$\mu$-recursive functions} (1933 at Vienna).
\item A few years later Alonzo Church and his student Stephen Kleene (1936 at Princeton) proposed the notion of \emph{$\lambda$-definable} functions and showed that many function that we consider computable are $\lambda$-definable. This system forms the basis of functional programming languages and automated theorem provers
  \item Another proposal was by Alan Turing (1936 at Cambridge/Princeton, 23 years old) in which he defined what we now call \emph{Turing machines}. This forms the basis of modern computers.
\end{itemize}
Soon after, Church, Kleene and Turing proved that their definitions are equivalent.
The subject `Theory of Computation' originated from the seminal results above. We list below some of the central ideas/themes of the subject as of today:
\begin{itemize}
\item Identify problems that are computable/uncomputable.
\item Classify computable problems on the basis of expressivity (dynamic memory,function calls, distributedness) and resource utilization (Time,Space, bandwidth).
\item Identify relationships between the above-mentioned classes. For example, can something that requires $n$ units of space, be done in $n^2$ units of time?
\item Study semantics of programming languages:
  \begin{itemize}
  \item Operational semantics defines the meaning of a program as its effect on the state of a machine.
  \item Denotational semantics defines the meaning of a program as the function it computes.
  \end{itemize}
\item Develop bug-free of programs via verification against specification (ee Hoare logic, Model cheching for more deatils) or synthesis from specification.
\end{itemize}

\end{document}
